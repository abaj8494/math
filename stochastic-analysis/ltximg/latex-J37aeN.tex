\documentclass[11pt]{article}
\usepackage[paperwidth=500pt,paperheight=5000pt,margin=10pt]{geometry}
\usepackage{fontspec}
\usepackage{amsmath,amssymb}
\usepackage[dvipsnames]{xcolor}
\usepackage{enumitem}
\usepackage{hyperref}
\hypersetup{colorlinks=false}
\setlength{\headheight}{13.59999pt}
\definecolor{linkblue}{HTML}{00007B}
\definecolor{exampleblue}{HTML}{4169E1}
\definecolor{highlightred}{HTML}{B22222}
\definecolor{highlightblue}{HTML}{0000FF}
\definecolor{notegray}{HTML}{7F807F}
\definecolor{highlightsalmon}{HTML}{F08080}
\definecolor{highlightgreen}{HTML}{006400}
\definecolor{examplebarbg}{rgb}{0.255, 0.41, 0.884}
\usepackage{amsthm}
\usepackage{mathtools}
\usepackage{thmtools}
\usepackage{mathrsfs}
\usepackage{bm}
\usepackage{fancyhdr}
\pagestyle{fancy}
\fancyhf{}
\lhead{\textit{Melantha Wang}}
\rhead{\thepage}
\renewcommand{\headrulewidth}{0pt}
\usepackage{tikz}
\usepackage{caption}
\usetikzlibrary{arrows.meta, positioning, shapes.geometric, calc}
\usepackage{mdframed}
\usepackage[bottom]{footmisc}
\setlength{\parskip}{0pt plus 1pt}
\usepackage{titlesec}
\titleformat{\section}{\normalfont\normalsize\centering}{\thesection}{1em}{\scshape}
\titleformat{\subsection}{\normalfont\bfseries\fontsize{12}{14.4}\selectfont}{\thesubsection}{1em}{}
\titleformat{\subsubsection}{\normalfont\bfseries}{\thesubsubsection}{1em}{}
\newtheoremstyle{italicbody}{3pt}{3pt}{\itshape}{}{\bfseries}{.}{ }{}
\newtheoremstyle{uprightbody}{3pt}{3pt}{}{}{\bfseries}{.}{ }{}
\theoremstyle{italicbody}
\newtheorem{theorem}{Theorem}[subsection]
\newtheorem{lemma}{Lemma}[subsection]
\newtheorem{corollary}{Corollary}[subsection]
\newtheorem{proposition}{Proposition}[subsection]
\theoremstyle{uprightbody}
\newtheorem{definition}{Definition}[subsection]
\newtheorem{examplesinner}{Examples}[subsection]
\newtheorem{remark}{Remark}[subsection]
\newtheorem{result}{Result}[subsection]
\newmdenv[leftline=true,rightline=false,topline=false,bottomline=false,linewidth=3pt,linecolor=examplebarbg,innerleftmargin=10pt,innerrightmargin=0pt,innertopmargin=2pt,innerbottommargin=2pt,skipabove=3pt,skipbelow=3pt]{examplebox}
\newenvironment{examples}[1][]{\begin{examplebox}\refstepcounter{examplesinner}{\color{exampleblue}\sffamily\bfseries Examples~\theexamplesinner\ifx\\#1\\\else\ (#1)\fi.}\,}{\end{examplebox}}
\renewcommand{\theHtheorem}{\thesection.\thesubsection.\arabic{theorem}}
\renewcommand{\theHlemma}{\thesection.\thesubsection.\arabic{lemma}}
\renewcommand{\theHcorollary}{\thesection.\thesubsection.\arabic{corollary}}
\renewcommand{\theHproposition}{\thesection.\thesubsection.\arabic{proposition}}
\renewcommand{\theHdefinition}{\thesection.\thesubsection.\arabic{definition}}
\renewcommand{\theHexamplesinner}{\thesection.\thesubsection.\arabic{examplesinner}}
\renewcommand{\theHremark}{\thesection.\thesubsection.\arabic{remark}}
\renewcommand{\theHresult}{\thesection.\thesubsection.\arabic{result}}
\newcommand{\E}{\mathbb{E}}
\newcommand{\Var}{\operatorname{Var}}
\newcommand{\Cov}{\operatorname{Cov}}
\renewcommand{\P}{\mathbb{P}}
\newcommand{\Q}{\mathbb{Q}}
\newcommand{\R}{\mathbb{R}}
\newcommand{\N}{\mathcal{N}}
\newcommand{\F}{\mathcal{F}}
\newcommand{\G}{\mathcal{G}}
\newcommand{\A}{\mathcal{A}}
\newcommand{\B}{\mathcal{B}}
\newcommand{\Sc}{\mathcal{S}^c}
\newcommand{\Lp}{\mathcal{L}}
\newcommand{\indic}{\mathbf{1}}
\DeclareMathOperator{\sgn}{sgn}
\newcommand{\hblue}[1]{{\color{highlightblue}#1}}
\newcommand{\hred}[1]{{\color{highlightred}#1}}
\newcommand{\hgray}[1]{{\color{notegray}#1}}
\newcommand{\hsalmon}[1]{{\color{highlightsalmon}#1}}
\newcommand{\hgreen}[1]{{\color{highlightgreen}#1}}
\renewcommand{\qedsymbol}{$\blacksquare$}
\pagestyle{empty}
\pagestyle{empty}
\begin{document}
ocal martingales]
\leavevmode
\begin{enumerate}[label=(\roman*)]
    \item Any non-negative $\F$-local martingale $M$ is an $\F$-supermartingale\footnote{The assumption that $M$ is non-negative can be replaced by the assumption that $M \geq \eta$ for some random variable $\eta$ with $\E(\eta) > -\infty$.}. If, in addition, $\E(M_t)$ is constant then $M$ is an $\F$-martingale.

    \item (corollary of above) Let $M$ be a non-negative $\F$-local martingale. If $M_0 = 0$ then $M_t = 0$ for every $t \in [0, T]$.

    \item (quadratic variation) Let $M$ be a continuous $\F$-local martingale. Then the quadratic variation $\langle M \rangle$ is the unique continuous, increasing and $\F$-adapted process with $\langle M \rangle_0 = 0$ such that the process $M^2 - \langle M \rangle$ is a continuous $\F$-local martingale.

    \begin{remark}
    The quadratic variation is invariant with respect to an $\F_0$-measurable shift of $M$; specifically, if $N = \psi + M$ for some $\F_0$-measurable random variable $\psi$ is a continuous local martingale, then $\langle N \rangle = \langle M \rangle$. In particular, $\langle M \rangle = \langle M - M_0 \rangle$.
    \end{remark}

    \item Let $M$ be a continuous $\F$-local martingale s.t.\ $\langle M \rangle_t = 0$ for $t \in [0, T]$. Then $M_t = M_0$ for every $t \in [0, T]$.

    \item Let $M$ be a continuous $\F$-local martingale of finite variation\footnote{An $\F$-adapted stochastic process $X = (X_t)_{t \in [0,T]}$ is said to be of finite variation if almost all sample paths of $X$ are functions of finite variation on $[0, T]$.}. Then $M_t = M_0$ for every $t \in [0, T]$.

    \item (corollary of above) If $M_0 = 0$ and $M$ is a continuous $\F$-local martingale of finite variation, then $M$ vanishes (more precisely, it is indistinguishable from the null process).
\end{enumerate}
\end{theorem}

\begin{proof}
(
\end{document}
