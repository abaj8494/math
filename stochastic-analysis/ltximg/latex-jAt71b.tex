\documentclass[11pt]{article}
\usepackage[paperwidth=500pt,paperheight=5000pt,margin=10pt]{geometry}
\usepackage{fontspec}
\usepackage{amsmath,amssymb}
\usepackage[dvipsnames]{xcolor}
\usepackage{enumitem}
\usepackage{hyperref}
\hypersetup{colorlinks=false}
\setlength{\headheight}{13.59999pt}
\definecolor{linkblue}{HTML}{00007B}
\definecolor{exampleblue}{HTML}{4169E1}
\definecolor{highlightred}{HTML}{B22222}
\definecolor{highlightblue}{HTML}{0000FF}
\definecolor{notegray}{HTML}{7F807F}
\definecolor{highlightsalmon}{HTML}{F08080}
\definecolor{highlightgreen}{HTML}{006400}
\definecolor{examplebarbg}{rgb}{0.255, 0.41, 0.884}
\usepackage{amsthm}
\usepackage{mathtools}
\usepackage{thmtools}
\usepackage{mathrsfs}
\usepackage{bm}
\usepackage{fancyhdr}
\pagestyle{fancy}
\fancyhf{}
\lhead{\textit{Melantha Wang}}
\rhead{\thepage}
\renewcommand{\headrulewidth}{0pt}
\usepackage{tikz}
\usepackage{caption}
\usetikzlibrary{arrows.meta, positioning, shapes.geometric, calc}
\usepackage{mdframed}
\usepackage[bottom]{footmisc}
\setlength{\parskip}{0pt plus 1pt}
\usepackage{titlesec}
\titleformat{\section}{\normalfont\normalsize\centering}{\thesection}{1em}{\scshape}
\titleformat{\subsection}{\normalfont\bfseries\fontsize{12}{14.4}\selectfont}{\thesubsection}{1em}{}
\titleformat{\subsubsection}{\normalfont\bfseries}{\thesubsubsection}{1em}{}
\newtheoremstyle{italicbody}{3pt}{3pt}{\itshape}{}{\bfseries}{.}{ }{}
\newtheoremstyle{uprightbody}{3pt}{3pt}{}{}{\bfseries}{.}{ }{}
\theoremstyle{italicbody}
\newtheorem{theorem}{Theorem}[subsection]
\newtheorem{lemma}{Lemma}[subsection]
\newtheorem{corollary}{Corollary}[subsection]
\newtheorem{proposition}{Proposition}[subsection]
\theoremstyle{uprightbody}
\newtheorem{definition}{Definition}[subsection]
\newtheorem{examplesinner}{Examples}[subsection]
\newtheorem{remark}{Remark}[subsection]
\newtheorem{result}{Result}[subsection]
\newmdenv[leftline=true,rightline=false,topline=false,bottomline=false,linewidth=3pt,linecolor=examplebarbg,innerleftmargin=10pt,innerrightmargin=0pt,innertopmargin=2pt,innerbottommargin=2pt,skipabove=3pt,skipbelow=3pt]{examplebox}
\newenvironment{examples}[1][]{\begin{examplebox}\refstepcounter{examplesinner}{\color{exampleblue}\sffamily\bfseries Examples~\theexamplesinner\ifx\\#1\\\else\ (#1)\fi.}\,}{\end{examplebox}}
\renewcommand{\theHtheorem}{\thesection.\thesubsection.\arabic{theorem}}
\renewcommand{\theHlemma}{\thesection.\thesubsection.\arabic{lemma}}
\renewcommand{\theHcorollary}{\thesection.\thesubsection.\arabic{corollary}}
\renewcommand{\theHproposition}{\thesection.\thesubsection.\arabic{proposition}}
\renewcommand{\theHdefinition}{\thesection.\thesubsection.\arabic{definition}}
\renewcommand{\theHexamplesinner}{\thesection.\thesubsection.\arabic{examplesinner}}
\renewcommand{\theHremark}{\thesection.\thesubsection.\arabic{remark}}
\renewcommand{\theHresult}{\thesection.\thesubsection.\arabic{result}}
\newcommand{\E}{\mathbb{E}}
\newcommand{\Var}{\operatorname{Var}}
\newcommand{\Cov}{\operatorname{Cov}}
\renewcommand{\P}{\mathbb{P}}
\newcommand{\Q}{\mathbb{Q}}
\newcommand{\R}{\mathbb{R}}
\newcommand{\N}{\mathcal{N}}
\newcommand{\F}{\mathcal{F}}
\newcommand{\G}{\mathcal{G}}
\newcommand{\A}{\mathcal{A}}
\newcommand{\B}{\mathcal{B}}
\newcommand{\Sc}{\mathcal{S}^c}
\newcommand{\Lp}{\mathcal{L}}
\newcommand{\indic}{\mathbf{1}}
\DeclareMathOperator{\sgn}{sgn}
\newcommand{\hblue}[1]{{\color{highlightblue}#1}}
\newcommand{\hred}[1]{{\color{highlightred}#1}}
\newcommand{\hgray}[1]{{\color{notegray}#1}}
\newcommand{\hsalmon}[1]{{\color{highlightsalmon}#1}}
\newcommand{\hgreen}[1]{{\color{highlightgreen}#1}}
\renewcommand{\qedsymbol}{$\blacksquare$}
\pagestyle{empty}
\pagestyle{empty}
\begin{document}
\begin{definition}[$\sigma$-algebra generated by families of sets]
\leavevmode
\begin{itemize}
    \item Let $\mathcal{C}$ be an arbitrary family of subsets of $\Omega$. We denote by $\sigma(\mathcal{C})$ the smallest $\sigma$-algebra which contains every set in $\mathcal{C}$ (i.e.\ $\mathcal{C} \subseteq \sigma(\mathcal{C})$) and call this the $\sigma$-algebra generated by $\mathcal{C}$.

    \item (Borel $\sigma$-algebra) An important example is the Borel $\sigma$-algebra over any topological space $\Omega$, denoted by $\B(\Omega)$, which is the $\sigma$-algebra generated by the open sets of $\Omega$ (or, equivalently, by the closed sets\footnote{To show this, we just need to show that $\sigma(\mathcal{C}) \subseteq \B(\Omega)$ and $\B(\Omega) \subseteq \sigma(\mathcal{C})$. All closed sets are complements of open sets. Since $\B(\Omega)$ being a $\sigma$-algebra is closed under complement, it contains all the closed sets, i.e.\ $\mathcal{C} \subseteq \B(\Omega) \Rightarrow \sigma(\mathcal{C}) \subseteq \B(\Omega)$. A similar argument can be used to show that $\mathcal{O} \subseteq \sigma(\mathcal{C}) \Rightarrow \B(\Omega) = \sigma(\mathcal{O}) \subseteq \sigma(\mathcal{C})$.}). In other words, $\B(\Omega) := \sigma(\mathcal{O}(\Omega))$, where $\mathcal{O}(\cdot)$ denotes the collection of all open sets.
    \begin{itemize}
        \item Recall that an open set of $\R$ is a subset $E \subseteq \R$ such that for every $x \in E$ there exists $\epsilon > 0$ such that $B_\epsilon(x) = \{y \in \R : |x - y| < \epsilon\}$ is contained in $E$.
        \item A set $F \subseteq \R$ is said to be closed if $F^c$ is open.
        \item $\R$ and $\emptyset$ are simultaneously both open and closed sets.
    \end{itemize}

    \item ($\sigma$-algebra generated by a random variable) Given $X : (\Omega, \A) \to (\Psi, \G)$, the $\sigma$-algebra generated by $X$, denoted $\sigma(X)$ is the smallest $\sigma$-algebra on $\Omega$ such that $X$ is a random variable, that is, $X$ is measurable with respect to $\sigma(X)$ and $\G$. Equivalently, $\sigma(X) = X^{-1}(\G) = \{X^{-1}(S) \mid S \in \G\}$ (by Definition~1.3.1 and Theorem~1.3.2).
\end{itemize}
\end{definition}
\end{document}
