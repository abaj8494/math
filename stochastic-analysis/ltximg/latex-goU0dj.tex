\documentclass[11pt]{article}
\usepackage[paperwidth=500pt,paperheight=5000pt,margin=10pt]{geometry}
\usepackage{fontspec}
\usepackage{amsmath,amssymb}
\usepackage[dvipsnames]{xcolor}
\usepackage{enumitem}
\usepackage{hyperref}
\hypersetup{colorlinks=false}
\setlength{\headheight}{13.59999pt}
\definecolor{linkblue}{HTML}{00007B}
\definecolor{exampleblue}{HTML}{4169E1}
\definecolor{highlightred}{HTML}{B22222}
\definecolor{highlightblue}{HTML}{0000FF}
\definecolor{notegray}{HTML}{7F807F}
\definecolor{highlightsalmon}{HTML}{F08080}
\definecolor{highlightgreen}{HTML}{006400}
\definecolor{examplebarbg}{rgb}{0.255, 0.41, 0.884}
\usepackage{amsthm}
\usepackage{mathtools}
\usepackage{thmtools}
\usepackage{mathrsfs}
\usepackage{bm}
\usepackage{fancyhdr}
\pagestyle{fancy}
\fancyhf{}
\lhead{\textit{Melantha Wang}}
\rhead{\thepage}
\renewcommand{\headrulewidth}{0pt}
\usepackage{tikz}
\usepackage{caption}
\usetikzlibrary{arrows.meta, positioning, shapes.geometric, calc}
\usepackage{mdframed}
\usepackage[bottom]{footmisc}
\setlength{\parskip}{0pt plus 1pt}
\usepackage{titlesec}
\titleformat{\section}{\normalfont\normalsize\centering}{\thesection}{1em}{\scshape}
\titleformat{\subsection}{\normalfont\bfseries\fontsize{12}{14.4}\selectfont}{\thesubsection}{1em}{}
\titleformat{\subsubsection}{\normalfont\bfseries}{\thesubsubsection}{1em}{}
\newtheoremstyle{italicbody}{3pt}{3pt}{\itshape}{}{\bfseries}{.}{ }{}
\newtheoremstyle{uprightbody}{3pt}{3pt}{}{}{\bfseries}{.}{ }{}
\theoremstyle{italicbody}
\newtheorem{theorem}{Theorem}[subsection]
\newtheorem{lemma}{Lemma}[subsection]
\newtheorem{corollary}{Corollary}[subsection]
\newtheorem{proposition}{Proposition}[subsection]
\theoremstyle{uprightbody}
\newtheorem{definition}{Definition}[subsection]
\newtheorem{examplesinner}{Examples}[subsection]
\newtheorem{remark}{Remark}[subsection]
\newtheorem{result}{Result}[subsection]
\newmdenv[leftline=true,rightline=false,topline=false,bottomline=false,linewidth=3pt,linecolor=examplebarbg,innerleftmargin=10pt,innerrightmargin=0pt,innertopmargin=2pt,innerbottommargin=2pt,skipabove=3pt,skipbelow=3pt]{examplebox}
\newenvironment{examples}[1][]{\begin{examplebox}\refstepcounter{examplesinner}{\color{exampleblue}\sffamily\bfseries Examples~\theexamplesinner\ifx\\#1\\\else\ (#1)\fi.}\,}{\end{examplebox}}
\renewcommand{\theHtheorem}{\thesection.\thesubsection.\arabic{theorem}}
\renewcommand{\theHlemma}{\thesection.\thesubsection.\arabic{lemma}}
\renewcommand{\theHcorollary}{\thesection.\thesubsection.\arabic{corollary}}
\renewcommand{\theHproposition}{\thesection.\thesubsection.\arabic{proposition}}
\renewcommand{\theHdefinition}{\thesection.\thesubsection.\arabic{definition}}
\renewcommand{\theHexamplesinner}{\thesection.\thesubsection.\arabic{examplesinner}}
\renewcommand{\theHremark}{\thesection.\thesubsection.\arabic{remark}}
\renewcommand{\theHresult}{\thesection.\thesubsection.\arabic{result}}
\newcommand{\E}{\mathbb{E}}
\newcommand{\Var}{\operatorname{Var}}
\newcommand{\Cov}{\operatorname{Cov}}
\renewcommand{\P}{\mathbb{P}}
\newcommand{\Q}{\mathbb{Q}}
\newcommand{\R}{\mathbb{R}}
\newcommand{\N}{\mathcal{N}}
\newcommand{\F}{\mathcal{F}}
\newcommand{\G}{\mathcal{G}}
\newcommand{\A}{\mathcal{A}}
\newcommand{\B}{\mathcal{B}}
\newcommand{\Sc}{\mathcal{S}^c}
\newcommand{\Lp}{\mathcal{L}}
\newcommand{\indic}{\mathbf{1}}
\DeclareMathOperator{\sgn}{sgn}
\newcommand{\hblue}[1]{{\color{highlightblue}#1}}
\newcommand{\hred}[1]{{\color{highlightred}#1}}
\newcommand{\hgray}[1]{{\color{notegray}#1}}
\newcommand{\hsalmon}[1]{{\color{highlightsalmon}#1}}
\newcommand{\hgreen}[1]{{\color{highlightgreen}#1}}
\renewcommand{\qedsymbol}{$\blacksquare$}
\pagestyle{empty}
\pagestyle{empty}
\begin{document}

\begin{examples}[Radon--Nikodym Density]
\leavevmode
\begin{enumerate}
    \item Coin flip with $\Omega = \{T, H\}$. Consider $\P$ given by $\P\{H\} = \P\{T\} = 1/2$ and $\Q$ by $\Q\{H\} = 1/3$, $\Q\{T\} = 2/3$. Clearly $\P$ and $\Q$ are equivalent; further,
    \begin{align*}
    \Q\{T\} &= \E^\P[\eta \indic_{\{T\}}] = \eta\{T\} \P\{T\} \\
    \Q\{H\} &= \E^\P[\eta \indic_{\{H\}}] = \eta\{H\} \P\{H\}
    \end{align*}
    $\implies \eta(\omega) = \frac{\Q(\omega)}{\P(\omega)}$, $\omega \in \{T, H\}$.

    \item Consider $\Omega = [0, \infty)$ and $\A = \B([0, \infty))$. One can define two equivalent probability measure on $(\Omega, \A)$ by setting, for every $A \in \A$,
    \begin{align*}
    \P(A) &= \int_\Omega \indic_A(x) \indic_{[0,1]}(x) \, dx = \int_0^\infty \indic_A(x) \, d\P(x) \\
    \Q(A) &= \int_\Omega \indic_A(x) e^{-x} \, dx = \int_0^\infty \indic_A(x) \, d\Q(x)
    \end{align*}
    The probability measure $\P$ is the uniform (Lebesgue) measure on $[0, 1]$ and $\Q$ is the exponential measure. It is not difficult to see that $\P$ is absolutely continuous with respect to $\Q$ but not the reverse, as $\P(A) \leq e \Q(A)$.

    To find the Radon--Nikodym density of $\Q$ with respect to $\P$ we note that for any $A \in \A$,
    \[
    \P(A) = \int_\Omega \indic_A(x) \indic_{[0,1]}(x) \, dx = \int_\Omega \indic_A(x) \frac{\indic_{[0,1]}(x)}{e^{-x}} e^{-x} \, dx = \int_\Omega \indic_A(x) \frac{\indic_{[0,1]}(x)}{e^{-x}} \, d\Q(x)
    \]
    Therefore the Radon--Nikodym density of $\P$ w.r.t.\ $\Q$ is given by $\eta(x) = \indic_{[0,1]}(x) / e^{-x}$. Further, if we let $X(\omega) = \omega$, then we can show that $X \sim U[0, 1]$ under $\P$ and $X \sim \text{Exp}(1)$ under $\Q$ by computing the CDFs.
\end{enumerate}
\end{examples
\end{document}
