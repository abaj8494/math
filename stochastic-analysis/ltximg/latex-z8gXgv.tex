\documentclass[11pt]{article}
\usepackage[paperwidth=500pt,paperheight=5000pt,margin=10pt]{geometry}
\usepackage{fontspec}
\usepackage{amsmath,amssymb}
\usepackage[dvipsnames]{xcolor}
\usepackage{enumitem}
\usepackage{hyperref}
\hypersetup{colorlinks=false}
\setlength{\headheight}{13.59999pt}
\definecolor{linkblue}{HTML}{00007B}
\definecolor{exampleblue}{HTML}{4169E1}
\definecolor{highlightred}{HTML}{B22222}
\definecolor{highlightblue}{HTML}{0000FF}
\definecolor{notegray}{HTML}{7F807F}
\definecolor{highlightsalmon}{HTML}{F08080}
\definecolor{highlightgreen}{HTML}{006400}
\definecolor{examplebarbg}{rgb}{0.255, 0.41, 0.884}
\usepackage{amsthm}
\usepackage{mathtools}
\usepackage{thmtools}
\usepackage{mathrsfs}
\usepackage{bm}
\usepackage{fancyhdr}
\pagestyle{fancy}
\fancyhf{}
\lhead{\textit{Melantha Wang}}
\rhead{\thepage}
\renewcommand{\headrulewidth}{0pt}
\usepackage{tikz}
\usepackage{caption}
\usetikzlibrary{arrows.meta, positioning, shapes.geometric, calc}
\usepackage{mdframed}
\usepackage[bottom]{footmisc}
\setlength{\parskip}{0pt plus 1pt}
\usepackage{titlesec}
\titleformat{\section}{\normalfont\normalsize\centering}{\thesection}{1em}{\scshape}
\titleformat{\subsection}{\normalfont\bfseries\fontsize{12}{14.4}\selectfont}{\thesubsection}{1em}{}
\titleformat{\subsubsection}{\normalfont\bfseries}{\thesubsubsection}{1em}{}
\newtheoremstyle{italicbody}{3pt}{3pt}{\itshape}{}{\bfseries}{.}{ }{}
\newtheoremstyle{uprightbody}{3pt}{3pt}{}{}{\bfseries}{.}{ }{}
\theoremstyle{italicbody}
\newtheorem{theorem}{Theorem}[subsection]
\newtheorem{lemma}{Lemma}[subsection]
\newtheorem{corollary}{Corollary}[subsection]
\newtheorem{proposition}{Proposition}[subsection]
\theoremstyle{uprightbody}
\newtheorem{definition}{Definition}[subsection]
\newtheorem{examplesinner}{Examples}[subsection]
\newtheorem{remark}{Remark}[subsection]
\newtheorem{result}{Result}[subsection]
\newmdenv[leftline=true,rightline=false,topline=false,bottomline=false,linewidth=3pt,linecolor=examplebarbg,innerleftmargin=10pt,innerrightmargin=0pt,innertopmargin=2pt,innerbottommargin=2pt,skipabove=3pt,skipbelow=3pt]{examplebox}
\newenvironment{examples}[1][]{\begin{examplebox}\refstepcounter{examplesinner}{\color{exampleblue}\sffamily\bfseries Examples~\theexamplesinner\ifx\\#1\\\else\ (#1)\fi.}\,}{\end{examplebox}}
\renewcommand{\theHtheorem}{\thesection.\thesubsection.\arabic{theorem}}
\renewcommand{\theHlemma}{\thesection.\thesubsection.\arabic{lemma}}
\renewcommand{\theHcorollary}{\thesection.\thesubsection.\arabic{corollary}}
\renewcommand{\theHproposition}{\thesection.\thesubsection.\arabic{proposition}}
\renewcommand{\theHdefinition}{\thesection.\thesubsection.\arabic{definition}}
\renewcommand{\theHexamplesinner}{\thesection.\thesubsection.\arabic{examplesinner}}
\renewcommand{\theHremark}{\thesection.\thesubsection.\arabic{remark}}
\renewcommand{\theHresult}{\thesection.\thesubsection.\arabic{result}}
\newcommand{\E}{\mathbb{E}}
\newcommand{\Var}{\operatorname{Var}}
\newcommand{\Cov}{\operatorname{Cov}}
\renewcommand{\P}{\mathbb{P}}
\newcommand{\Q}{\mathbb{Q}}
\newcommand{\R}{\mathbb{R}}
\newcommand{\N}{\mathcal{N}}
\newcommand{\F}{\mathcal{F}}
\newcommand{\G}{\mathcal{G}}
\newcommand{\A}{\mathcal{A}}
\newcommand{\B}{\mathcal{B}}
\newcommand{\Sc}{\mathcal{S}^c}
\newcommand{\Lp}{\mathcal{L}}
\newcommand{\indic}{\mathbf{1}}
\DeclareMathOperator{\sgn}{sgn}
\newcommand{\hblue}[1]{{\color{highlightblue}#1}}
\newcommand{\hred}[1]{{\color{highlightred}#1}}
\newcommand{\hgray}[1]{{\color{notegray}#1}}
\newcommand{\hsalmon}[1]{{\color{highlightsalmon}#1}}
\newcommand{\hgreen}[1]{{\color{highlightgreen}#1}}
\renewcommand{\qedsymbol}{$\blacksquare$}
\pagestyle{empty}
\pagestyle{empty}
\begin{document}
\begin{proof}
For (iii), we prove the case for an elementary process $\gamma \in \mathcal{R}$. Since the expected value of the cross terms are zero, i.e.\ $\E[(W_{t_{j+1}} - W_{t_j})(W_{t_{k+1}} - W_{t_k})] = 0$ for $j \neq k$, we obtain
\begin{align*}
\E\left[\left(\sum_{j=0}^{m-1} \gamma_j (W_{t_{j+1}} - W_{t_j})\right)^2\right]
&= \E\left[\sum_{j=0}^{m-1} \gamma_j^2 (W_{t_{j+1}} - W_{t_j})^2\right] \\
&= \E\left[\sum_{i=0}^{m-1} \gamma_j^2 \E\left[(W_{t_{j+1}} - W_{t_j})^2 \mid \F_{t_j}\right]\right] \\
&= \E\left[\sum_{i=0}^{m-1} \gamma_j^2 (t_{j+1} - t_j)\right] \\
&= \|\gamma\|^2_W
\end{align*}
This means that $\hat{I}_T : (\mathcal{K}, \|\cdot\|_W) \to L^2(\Omega, \F_T, \P)$ is an isometry, i.e.\ a distance preserving transformation. This can be extended to the isometry $I_T : (L^2_\P(W), \|\cdot\|_W) \to L^2(\Omega, \F_T, \P)$.\footnote{The extension is done using the fact that the class of simple process $\mathcal{K}$ is a dense subset in the Banach space $L^2_\P(W)$ and the space $L^2(\Omega, \F_T, \P)$ is also complete (since it a Banach space or more specifically a Hilbert space).}

For (iv), again we consider an elementary process $\gamma$ and we can then do a similar proof to that of Theorem~3.3.2. We first compute the quadratic variation accumulated by the Ito integral on one of the subintervals $[t_j, t_{j+1}]$ on which $\gamma_u = \gamma_j$ is constant. For this, we choose partition points
\[
t_j = s_0 < s_1 < \cdots < s_m = t_{j+1}
\]
and consider
\[
\sum_{i=0}^{m-1} (I_{s_{i+1}}(\gamma) - I_{s_i}(\gamma))^2 = \sum_{i=0}^{m-1} (\gamma_j (W_{s_{i+1}} - W_{s_i}))^2 = \gamma_j^2 \sum_{i=0}^{m-1} (W_{s_{i+1}} - W_{s_i})^2
\]
As $m \to \infty$ and the step size $\max_{0 \leq i \leq m-1} (s_{i+1} - s_i)$ approaches zero, the term $\sum_{i=0}^{m-1} (W_{s_{i+1}} - W_{s_i})^2$ converges to the quadratic variation accumulated by Brownian motion between times $t_j$ and $t_{j+1}$, which is $t_{j+1} - t_j$. Therefore, the limit of the above is
\[
\gamma_j^2 (t_{j+1} - t_j) = \int_{t_j}^{t_{j+1}} \gamma_u^2 \, du.
\]
Adding up all these pieces for each of the subintervals $[t_j, t_{j+1}]$ completes the proof.
\end{proof}
\end{document}
