\documentclass[11pt]{article}
\usepackage[paperwidth=500pt,paperheight=5000pt,margin=10pt]{geometry}
\usepackage{fontspec}
\usepackage{amsmath,amssymb}
\usepackage[dvipsnames]{xcolor}
\usepackage{enumitem}
\usepackage{hyperref}
\hypersetup{colorlinks=false}
\setlength{\headheight}{13.59999pt}
\definecolor{linkblue}{HTML}{00007B}
\definecolor{exampleblue}{HTML}{4169E1}
\definecolor{highlightred}{HTML}{B22222}
\definecolor{highlightblue}{HTML}{0000FF}
\definecolor{notegray}{HTML}{7F807F}
\definecolor{highlightsalmon}{HTML}{F08080}
\definecolor{highlightgreen}{HTML}{006400}
\definecolor{examplebarbg}{rgb}{0.255, 0.41, 0.884}
\usepackage{amsthm}
\usepackage{mathtools}
\usepackage{thmtools}
\usepackage{mathrsfs}
\usepackage{bm}
\usepackage{fancyhdr}
\pagestyle{fancy}
\fancyhf{}
\lhead{\textit{Melantha Wang}}
\rhead{\thepage}
\renewcommand{\headrulewidth}{0pt}
\usepackage{tikz}
\usepackage{caption}
\usetikzlibrary{arrows.meta, positioning, shapes.geometric, calc}
\usepackage{mdframed}
\usepackage[bottom]{footmisc}
\setlength{\parskip}{0pt plus 1pt}
\usepackage{titlesec}
\titleformat{\section}{\normalfont\normalsize\centering}{\thesection}{1em}{\scshape}
\titleformat{\subsection}{\normalfont\bfseries\fontsize{12}{14.4}\selectfont}{\thesubsection}{1em}{}
\titleformat{\subsubsection}{\normalfont\bfseries}{\thesubsubsection}{1em}{}
\newtheoremstyle{italicbody}{3pt}{3pt}{\itshape}{}{\bfseries}{.}{ }{}
\newtheoremstyle{uprightbody}{3pt}{3pt}{}{}{\bfseries}{.}{ }{}
\theoremstyle{italicbody}
\newtheorem{theorem}{Theorem}[subsection]
\newtheorem{lemma}{Lemma}[subsection]
\newtheorem{corollary}{Corollary}[subsection]
\newtheorem{proposition}{Proposition}[subsection]
\theoremstyle{uprightbody}
\newtheorem{definition}{Definition}[subsection]
\newtheorem{examplesinner}{Examples}[subsection]
\newtheorem{remark}{Remark}[subsection]
\newtheorem{result}{Result}[subsection]
\newmdenv[leftline=true,rightline=false,topline=false,bottomline=false,linewidth=3pt,linecolor=examplebarbg,innerleftmargin=10pt,innerrightmargin=0pt,innertopmargin=2pt,innerbottommargin=2pt,skipabove=3pt,skipbelow=3pt]{examplebox}
\newenvironment{examples}[1][]{\begin{examplebox}\refstepcounter{examplesinner}{\color{exampleblue}\sffamily\bfseries Examples~\theexamplesinner\ifx\\#1\\\else\ (#1)\fi.}\,}{\end{examplebox}}
\renewcommand{\theHtheorem}{\thesection.\thesubsection.\arabic{theorem}}
\renewcommand{\theHlemma}{\thesection.\thesubsection.\arabic{lemma}}
\renewcommand{\theHcorollary}{\thesection.\thesubsection.\arabic{corollary}}
\renewcommand{\theHproposition}{\thesection.\thesubsection.\arabic{proposition}}
\renewcommand{\theHdefinition}{\thesection.\thesubsection.\arabic{definition}}
\renewcommand{\theHexamplesinner}{\thesection.\thesubsection.\arabic{examplesinner}}
\renewcommand{\theHremark}{\thesection.\thesubsection.\arabic{remark}}
\renewcommand{\theHresult}{\thesection.\thesubsection.\arabic{result}}
\newcommand{\E}{\mathbb{E}}
\newcommand{\Var}{\operatorname{Var}}
\newcommand{\Cov}{\operatorname{Cov}}
\renewcommand{\P}{\mathbb{P}}
\newcommand{\Q}{\mathbb{Q}}
\newcommand{\R}{\mathbb{R}}
\newcommand{\N}{\mathcal{N}}
\newcommand{\F}{\mathcal{F}}
\newcommand{\G}{\mathcal{G}}
\newcommand{\A}{\mathcal{A}}
\newcommand{\B}{\mathcal{B}}
\newcommand{\Sc}{\mathcal{S}^c}
\newcommand{\Lp}{\mathcal{L}}
\newcommand{\indic}{\mathbf{1}}
\DeclareMathOperator{\sgn}{sgn}
\newcommand{\hblue}[1]{{\color{highlightblue}#1}}
\newcommand{\hred}[1]{{\color{highlightred}#1}}
\newcommand{\hgray}[1]{{\color{notegray}#1}}
\newcommand{\hsalmon}[1]{{\color{highlightsalmon}#1}}
\newcommand{\hgreen}[1]{{\color{highlightgreen}#1}}
\renewcommand{\qedsymbol}{$\blacksquare$}
\pagestyle{empty}
\pagestyle{empty}
\begin{document}
\begin{examples}[Ito's lemma, 1D]
We can use Ito formula to determine whether a process is a martingale (no drift), submartingale (drift $> 0$) or supermartingale (drift $< 0$).

\begin{enumerate}
    \item $f(W_t) = (W_t)^2$. Then we have for all $t \in [0,T]$,
    \[
    df(W_t) = 2W_t \, dW_t + \frac{1}{2} \cdot 2 \, dt \implies W_t^2 = W_0^2 + \int_0^t 2W_s \, dW_s + \int_0^t ds = \int_0^t 2W_s \, dW_s + t.
    \]
    This also gives the Doob--Meyer decomposition (Theorem~2.4.2) of the positive submartingale $W^2$. Also recall that $W_t^2 - t = \int_0^t 2W_s \, dW_s$ is a martingale (e.g.\ Theorem~3.3.3, or Example~4.1.2).

    \item $Y_t = g(t, W_t) = e^{\mu t + W_t}$. Let $g(t, x) = e^{\mu t + x}$ and then $g_t = \mu Y_t$, $g_x = Y_t$, $g_{xx} = Y_t$ and so
    \begin{align*}
    dY_t &= \mu Y_t \, dt + Y_t \, dW_t + \frac{1}{2} \cdot Y_t \, d\langle W \rangle_t = \left(\mu + \frac{1}{2}\right) Y_t \, dt + Y_t \, dW_t \\
    &\implies Y_t = Y_0 + \int_0^t \underbrace{\left(\mu + \frac{1}{2}\right) Y_u \, du}_{\text{increasing}} + \int_0^t \underbrace{Y_u \, dW_u}_{\text{a martingale, as } Y \in L^2_\P(W)}.
    \end{align*}
    Therefore, we conclude that $Y$ is a submartingale.

    \item (integration by parts, Corollary~4.4.1) $Z_t = \sin(W_t) \cos(W_t)$. Then
    \[
    dZ_t = \sin W_t \, d(\cos W_t) + \cos W_t \, d(\sin W_t) + d\langle \sin W, \cos W \rangle_t.
    \]
    By Ito's lemma,
    \begin{align*}
    d(\cos W_t) &= -\sin W_t \, dW_t - \frac{1}{2} \cos W_t \, dt, \\
    d(\sin W_t) &= \cos W_t \, dW_t - \frac{1}{2} \sin W_t \, dt.
    \end{align*}
    Also $d\langle \sin W, \cos W \rangle_t = -\sin W_t \cos W_t \, dt$, so
    \[
    dZ_t = -\sin^2(W_t) \, dW_t + \cos^2(W_t) \, dW_t - 2 \sin W_t \cos W_t \, dt
    \]
    The sign of the drift term is not always positive or negative, and therefore $Z$ is none of a martingale, submartingale or supermartingale.
\end{enumerate}
\end{examples}
\end{document}
