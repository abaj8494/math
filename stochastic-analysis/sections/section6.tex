\section{Stochastic Differential Equations}

By a solution of the stochastic differential equation
\[
dX_t = \mu(t, X_t) \, dt + \sigma(t, X_t) \cdot dW_t
\]
with the initial condition $X_0$ given as an $\F_0$-measurable r.v., we mean an $\R^k$-valued, $\F$-adapted stochastic process $X$ defined on the probability space $(\Omega, \F, \P)$ and such that, for every $t \in [0, T]$,
\[
X_t = X_0 + \int_0^t \mu(u, X_u) \, du + \int_0^t \sigma(u, X_u) \cdot dW_u.
\]
Note that any solution $X$ to the SDE is an It\^o process.

\begin{definition}[pathwise uniqueness]
We say that the \textbf{pathwise uniqueness} of solutions to the above SDE holds if for any filtered probability space $(\Omega, \F, \P)$, any $d$-dimensional standard Brownian motions $W$ and $\widetilde{W}$ defined on $(\Omega, \F, \P)$, and any two solutions $X$ and $\widetilde{X}$ driven by $W$ and $\widetilde{W}$ respectively, the following implication is true
\[
\P\{W_t = \widetilde{W}_t \mid \forall t \in [0, T]\} = 1 \implies \P\{X_t = \widetilde{X}_t \mid \forall t \in [0, T]\} = 1.
\]
\end{definition}

\begin{remark}
The pathwise uniqueness of solutions is sometimes referred to as the strong uniqueness. This is due to the fact that under pathwise uniqueness any solution to the SDE is strong, meaning that it is adapted to the filtration generated by the driving Brownian motion $W$, $\F^W$.
\end{remark}

\subsection{It\^o's Existence and Uniqueness Theorem}

\begin{theorem}[It\^o's theorem, sufficient but not necessary conditions for the uniqueness of SDE solution]
Let $\mu : [0, T] \times \R \to \R$ and $\sigma : [0, T] \times \R \to \R$ satisfy the following conditions:
\begin{enumerate}[(i)]
    \item (Lipschitz continuity) $\mu$ and $\sigma$ are Lipschitz continuous with respect to the variable $x$, that is, there exist constants $K_1, K_2 > 0$ such that, for any $x, y \in \R$ and $t \in \R_+$,
    \begin{align*}
    |\mu(t, x) - \mu(t, y)| &\leq K_1 |x - y| \\
    |\sigma(t, x) - \sigma(t, y)| &\leq K_2 |x - y|
    \end{align*}

    \item (linear growth condition) $\mu$ and $\sigma$ satisfy the linear growth condition, i.e.\ there exist constants $C_1, C_2 > 0$ such that, for any $x \in \R$ and $t \in \R_+$,
    \begin{align*}
    |\mu(t, x)| &\leq C_1(1 + |x|) \\
    |\sigma(t, x)| &\leq C_2(1 + |x|)
    \end{align*}
\end{enumerate}
Then the SDE has a unique solution $X$.
\end{theorem}

\begin{remark}
To check Lipschitz continuity, it is sufficient to check that $\mu$ and $\sigma$ have bounded derivative. By Mean Value Theorem,
\[
\mu(t, x) - \mu(t, y) = \int_y^x \mu'(t, z) \, dz
\]
so if $|\mu'| \leq K$, then
\[
|\mu(t, x) - \mu(t, y)| \leq K \left|\int_y^x dz\right| = K|x - y|.
\]
\end{remark}

\subsection{Linear SDE}

\begin{proposition}
The unique solution of the SDE
\[
dX_t = \mu(t, X_t) \, dt + \sigma(t, X_t) \cdot d\bm{W}_t
\]
with
\begin{align*}
\mu(t, X_t) &= A_t X_t + a_t, \\
\sigma(t, X_t) &= [B_t^1 X_t + b_t^1, \cdots, B_t^d X_t + b_t^d],
\end{align*}
where $A$, $a$, $B^i$, $b^i$ $(i = 1, \cdots, d)$ are all $\F$-adapted bounded processes, is given by the formula
\[
X_t = \Phi_t \left(X_0 + \int_0^t \Phi_u^{-1} [a_u - \bm{B}_u \cdot \bm{b}_u] \, du + \int_0^t \Phi_u^{-1} \bm{b}_u \cdot d\bm{W}_u\right)
\]
where
\[
\Phi_t = \exp\left(\int_0^t \left(A_u - \frac{\bm{B}_u \cdot \bm{B}_u}{2}\right) du + \int_0^t \bm{B}_u \cdot d\bm{W}_u\right).
\]
\end{proposition}

\begin{remark}
Analogous to ODE, we can consider the integration factor given by
\[
Y_t = \Phi_t^{-1} = \exp\left(-\int_0^t A_u \, du - \int_0^t \bm{B}_u \cdot d\bm{W}_u + \frac{1}{2} \int_0^t \bm{B}_u \cdot \bm{B}_u \, du\right)
\]
\end{remark}

\begin{proof}
To check that $X$ is a solution to the SDE, it suffices to differentiate the RHS using It\^o's formula (easier: compute $d(X_t Y_t)$). The uniqueness of solution can be deduced from It\^o's Theorem~(6.1.1).
\end{proof}

\subsection{Non-Linear SDE}

Solving non-linear SDEs requires us to apply some suitable transform to the original equation.

\begin{examples}[Non-linear SDE: CIR process]
We solve the SDE
\[
dX_t = \left(\frac{1}{4} - bX_t\right) dt + \sqrt{X_t} \, dW_t, \quad X_0 = 1
\]
by considering the $Y_t := \sqrt{X_t}$. We apply It\^o's lemma to $Y_t := \sqrt{X_t}$ (with the omission of some technicalities).
\begin{align*}
dY_t &= \frac{1}{2\sqrt{X_t}} \, dX_t + \frac{1}{2} \cdot \left(-\hblue{\frac{1}{4}} \hblue{\frac{1}{X_t^{3/2}}}\right) d\langle X \rangle_t \\
&= \frac{1}{2\sqrt{X_t}} \left[\left(\frac{1}{4} - bX_t\right) dt + \sqrt{X_t} \, dW_t\right] - \hblue{\frac{1}{8}} \hblue{\frac{1}{X_t^{3/2}}} X_t \, dt \\
&= -\frac{b\sqrt{X_t}}{2} \, dt + \frac{1}{2} \, dW_t =: -\frac{b}{2} Y_t \, dt + \frac{1}{2} \, dW_t
\end{align*}
\ldots and we know how to solve linear SDEs! By applying an integration factor of $e^{bt/2}$ or otherwise, we can show
\[
Y_t = e^{-bt/2} + \frac{1}{2} \int_0^t e^{-\frac{1}{2}b(t-s)} \, dW_s
\]
and therefore the solution to the SDE is given by
\[
X_t = \left(e^{-bt/2} + \frac{1}{2} \int_0^t e^{-\frac{1}{2}b(t-s)} \, dW_s\right)^2.
\]
\end{examples}

\subsection{Stochastic Exponential and Stochastic Logarithmic}

Recall the definition of an It\^o's integral: Let $W$ be a $d$-dimensional standard Brownian motion defined on a filtered probability space $(\Omega, \F, \P)$. For an $\R^d$-valued process $\gamma \in \Lp_\P(W)$, we define the real-valued $\F$-adapted process $U$ by setting
\[
U_t = I_t(\gamma) = \int_0^t \gamma_u \cdot d\bm{W}_u, \quad t \in [0, T]
\]
The process $U$ defined in this way is, of course, a continuous $\F$-local martingale.

\begin{definition}[stochastic exponential]
The \textbf{stochastic exponential} of $U$ is given by the formula
\[
\mathcal{E}_t(U) = \mathcal{E}_t\left(\int_0^{\cdot} \gamma_u \cdot d\bm{W}_u\right) = \exp\left(\int_0^t \gamma_u \cdot d\bm{W}_u - \frac{1}{2} \int_0^t |\gamma_u|^2 \, du\right), \quad t \in [0, T]
\]
that is, $\mathcal{E}_t(U) = \exp(U_t - \langle U \rangle_t / 2)$. More generally, for any continuous local martingale $M$, we define
\[
\mathcal{E}_t(M) = \exp\left(M_t - \frac{1}{2}\langle M \rangle_t\right), \quad t \in [0, T].
\]
\end{definition}

\begin{lemma}
The stochastic exponential of $U$ is the unique solution $X$ of the stochastic differential equation
\[
dX_t = X_t \gamma_t \cdot d\bm{W}_t = X_t \, dU_t
\]
with the initial condition $X_0 = 1$. More generally, for any continuous local martingale $M$, the stochastic exponential $\mathcal{E}(M)$ is the unique solution $X$ to $dX_t = X_t \, dM_t$ with the initial condition $X_0 = 1$.
\end{lemma}

\begin{remark}
\leavevmode
\begin{enumerate}
    \item It follows immediately that $d\mathcal{E}_t(U) = \mathcal{E}_t(U) \gamma_t \cdot d\bm{W}_t = \mathcal{E}_t(U) \, dU_t$.

    \item Note that $\mathcal{E}(U)$ is a strictly positive continuous local martingale under $\P$ and thus, it follows a supermartingale with respect to $\F$. If, in addition,
    \[
    \E^\P(\mathcal{E}_T(U)) = \E^\P(\mathcal{E}_0(U)) = 1
    \]
    so that $\E^\P(\mathcal{E}_t(U))$ is constant on $t \in [0, T]$\footnote{For a supermartingale, we have $\E(M_T) \leq \E(M_t) \leq \E(M_0)$ for all $t \in [0, T]$, hence the ``so that'' conclusion.}, then the process $\mathcal{E}(U)$ is a continuous $\F$-martingale.
\end{enumerate}
\end{remark}

\begin{definition}[stochastic logarithm]
Given a continuous, strictly positive process $U$, the \textbf{stochastic logarithm} of $U$, denoted by $\mathcal{L}_t(U)$ is the (unique) solution to
\[
d\mathcal{L}_t(U) = \frac{1}{U_t} \, dU_t
\]
with $\mathcal{L}_0(U) = 0$.
\end{definition}

\begin{remark}
\leavevmode
\begin{enumerate}
    \item One can check that
    \[
    \mathcal{L}_t(U) = \int_0^t \frac{1}{U_s} \, dU_s = \log U_t + \frac{1}{2} \int_0^t \frac{1}{U_s^2} \, d\langle U \rangle_s
    \]
    is one (and ``the'') solution, by noting that $d\log U_t = \frac{1}{U_t} \, dU_t - \frac{1}{2} \frac{1}{U_t^2} \, d\langle U \rangle_t$.

    \item Stochastic logarithm is an inverse operation to stochastic exponential: $\mathcal{E}(\mathcal{L}(U)) = \mathcal{L}(\mathcal{E}(U)) = U$. Check
    \begin{align*}
    d\mathcal{L}_t(\mathcal{E}(U)) &= \frac{1}{\mathcal{E}_t(U)} \, d\mathcal{E}_t(U) = \frac{1}{\mathcal{E}_t(U)} \mathcal{E}_t(U) \, dU_t = dU_t, \\
    d\mathcal{E}_t(\mathcal{L}(U)) &= \mathcal{E}_t(\mathcal{L}(U)) \, d\mathcal{L}_t(U) = U_t \frac{1}{U_t} \, dU_t = dU_t.
    \end{align*}
\end{enumerate}
\end{remark}
