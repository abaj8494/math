\documentclass{article}
\usepackage{amsmath, amssymb, amsthm}
\usepackage{enumitem}

\newtheorem{theorem}{Theorem}
\newtheorem{definition}{Definition}
\newtheorem{lemma}{Lemma}
\newtheorem{corollary}{Corollary}
\newtheorem{proposition}{Proposition}

\begin{document}

\section*{Metric Spaces and Continuity}

\begin{definition}
A \emph{metric space} is a pair $(X, d)$, where $X$ is a non-empty set and $d : X \times X \to [0, \infty)$ is a function such that for all $x, y, z \in X$:
\begin{itemize}
  \item $d(x, y) = 0$ iff $x = y$
  \item $d(x, y) = d(y, x)$
  \item $d(x, z) \leq d(x, y) + d(y, z)$ (triangle inequality)
\end{itemize}
\end{definition}

\begin{definition}
A \emph{sequence} in a set $X$ is a function from $\mathbb{N}$ (or $\mathbb{Z}^+$) to $X$.
\end{definition}

\begin{theorem}
A sequence in a metric space can have at most one limit.
\end{theorem}

\begin{definition}
For a point $x$ in a metric space $(X, d)$ and $\varepsilon > 0$, the \emph{open $\varepsilon$-ball} is
\[
B(x, \varepsilon) = \{ y \in X : d(y, x) < \varepsilon \}.
\]
\end{definition}

\begin{definition}
Let $Y \subseteq X$ in a metric space $(X, d)$. Define:
\begin{itemize}
  \item $\mathrm{Int}(Y) = \{ y \in Y : \exists \varepsilon > 0 \text{ such that } B(y, \varepsilon) \subseteq Y \}$.
  \item $\mathrm{Bd}(Y) = X \setminus (\mathrm{Int}(Y) \cup \mathrm{Int}(X \setminus Y))$.
\end{itemize}
\end{definition}

\begin{definition}
$Y$ is \emph{open} if $Y = \mathrm{Int}(Y)$.
\end{definition}

\begin{definition}
$Y$ is \emph{closed} if $X \setminus Y$ is open.
\end{definition}

\begin{lemma}
Let $(X, d)$ be a metric space and $Y \subseteq X$. Then $\mathrm{Int}(\mathrm{Int}(Y)) = \mathrm{Int}(Y)$.
\end{lemma}

\begin{corollary}
$\mathrm{Int}(Y)$ is open.
\end{corollary}

\begin{definition}
The \emph{closure} of $Y$ is $\mathrm{Cl}(Y) = \mathrm{Int}(Y) \cup \mathrm{Bd}(Y)$.
\end{definition}

\begin{definition}
$Y$ is \emph{dense} if $\mathrm{Cl}(Y) = X$.
\end{definition}

\begin{definition}
A \emph{neighborhood} of $x$ is a set $U \subseteq X$ such that there exists an open set $V$ with $x \in V \subseteq U$.
\end{definition}

\begin{definition}
The set of open subsets of $X$ is called the \emph{topology} $\mathcal{O}(X)$.
\end{definition}

\begin{theorem}
The topology $\mathcal{O}(X)$ satisfies:
\begin{itemize}
  \item $\emptyset, X \in \mathcal{O}(X)$
  \item Arbitrary unions of open sets are open
  \item Finite intersections of open sets are open
\end{itemize}
\end{theorem}

\begin{definition}
Let $(X, d_X)$ and $(Y, d_Y)$ be metric spaces. A function $f: X \to Y$ is \emph{continuous} if for every open $V \subseteq Y$, the preimage $f^{-1}(V)$ is open in $X$.
\end{definition}

\begin{theorem}
If $f: X \to Y$ and $g: Y \to Z$ are continuous, then $g \circ f: X \to Z$ is continuous.
\end{theorem}

\begin{definition}
A subset $Y \subseteq X$ is \emph{bounded} if there exists $R > 0$ and $x \in X$ such that $Y \subseteq B(x, R)$.
\end{definition}

\begin{definition}
A sequence $\{x_n\}$ in $(X, d)$ is a \emph{Cauchy sequence} if for all $\varepsilon > 0$, there exists $N$ such that $d(x_m, x_n) < \varepsilon$ for all $m, n > N$.
\end{definition}

\begin{definition}
A metric space is \emph{complete} if every Cauchy sequence converges to a point in the space.
\end{definition}

\begin{theorem}
Let $(X, d)$ be a complete metric space. A subset $Y \subseteq X$ is complete iff $Y$ is closed.
\end{theorem}

\begin{definition}
Two Cauchy sequences $\{a_n\}$ and $\{b_n\}$ are equivalent if $\lim d(a_n, b_n) = 0$.
\end{definition}

\begin{definition}
The \emph{completion} of a metric space $(X, d)$ is the space of equivalence classes of Cauchy sequences with distance
\[
d([\{a_n\}], [\{b_n\}]) = \lim d(a_n, b_n).
\]
\end{definition}

\begin{theorem}
The completion $\overline{X}$ of $X$ is a complete metric space. The map $x \mapsto [\{x\}]$ is an isometry, and its image is dense in $\overline{X}$. The completion is unique up to isometric bijection.
\end{theorem}

\section*{Normed and Inner Product Spaces}

\begin{definition}
A \emph{norm} on a vector space $V$ is a function $\|\cdot\|: V \to [0, \infty)$ satisfying:
\begin{itemize}
  \item $\|x\| = 0$ iff $x = 0$
  \item $\|\lambda x\| = |\lambda| \cdot \|x\|$
  \item $\|x + y\| \leq \|x\| + \|y\|$
\end{itemize}
\end{definition}

\begin{theorem}
Let $(V, \|\cdot\|)$ be a normed vector space. Then $d(x, y) = \|x - y\|$ defines a metric.
\end{theorem}

\begin{definition}
A \emph{Banach space} is a complete normed vector space.
\end{definition}

\begin{definition}
For $p \in [1, \infty)$, define
\[
\ell^p = \left\{ \{x_n\} \subseteq \mathbb{R} : \sum_{n=1}^\infty |x_n|^p < \infty \right\},
\quad \text{with } \|x\|_p = \left( \sum |x_n|^p \right)^{1/p}.
\]
\end{definition}

\begin{theorem}
$(\ell^p, \|\cdot\|_p)$ is a Banach space.
\end{theorem}

\begin{definition}
An \emph{inner product space} is a vector space $V$ with a function $\langle \cdot, \cdot \rangle$ such that:
\begin{itemize}
  \item $\langle x, x \rangle > 0$ if $x \neq 0$
  \item $\langle x, y \rangle = \langle y, x \rangle$
  \item $\langle x + \lambda y, z \rangle = \langle x, z \rangle + \lambda \langle y, z \rangle$
\end{itemize}
\end{definition}

\begin{definition}
A \emph{Hilbert space} is a complete inner product space.
\end{definition}

\section*{Contraction and Lipschitz Mappings}

\begin{definition}
A \emph{contraction} is a function $f: X \to X$ such that there exists $c < 1$ with $d(f(x), f(y)) \leq c d(x, y)$.
\end{definition}

\begin{lemma}
Let $(X, d)$ be a metric space and $f$ a contraction. Then the sequence $x_{n+1} = f(x_n)$ is Cauchy.
\end{lemma}

\begin{theorem}[Contraction Mapping Theorem]
Let $(X, d)$ be a complete metric space and $f: X \to X$ a contraction. Then $f$ has a unique fixed point. Moreover, for any $x \in X$, the sequence $x_{n+1} = f(x_n)$ converges to that fixed point.
\end{theorem}

\begin{definition}
A function $f: X \to \mathbb{R}$ is \emph{Lipschitz continuous} if there exists $K > 0$ such that $|f(x) - f(y)| \leq K|x - y|$.
\end{definition}

\begin{definition}
A function $f: X \subseteq \mathbb{R}^2 \to \mathbb{R}$ is \emph{Lipschitz in the second variable} if
\[
|f(x, y_1) - f(x, y_2)| \leq K |y_1 - y_2|.
\]
\end{definition}

\begin{theorem}[Picard–Lindelöf Theorem]
Let $g$ be continuous near $(a, b) \in \mathbb{R}^2$ and Lipschitz in the second variable. Then the differential equation
\[
y' = g(x, y), \quad y(a) = b
\]
has a unique solution near $a$.
\end{theorem}

\end{document}
